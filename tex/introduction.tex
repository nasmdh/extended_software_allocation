\section{Introduction}
Over the last decades, the complexity of automotive functionality has increased tremendously, that is, the number of automotive functions (or software applications) has increased. Likewise some applications are computationally intensive, e.g., the computer-vision detection in self-driving vehiles using deep learning. Thus, there is a need for powerful computing architectures that would accommodate the current and future demands of software applications for computation. The distributed computing in automotive systems enables deployment of automotive application on multiple units to realize functionality, e.g., end-to-end behavior, which have timing constraints. Moreover, consolidation of applications on the same execution platform for efficiency has gained interest in the automotive domain, also known as \textit{mixed-criticality} design, that is after successful use cases in the avioncs. The distributed computing of applications and mixed-criticality design are interesting phenomena in the evolution automotive systems design, and in the context of software allocation, they come with two important challengs: i) real-time analysis is complex in distributed environement due to independently executing function on the different units. Considering independent failures of the computing unit, the distriburted architecture provides an opportunity to improve the reliability of the software applications by replicating functionality on multiple units; ii) the allocation of distributed software applciations is normally NP hard, and therefore finding optimal solutions is exponential. In this paper, we provide an integral software allocation approach basesd on metaheuristics that considers the timing and reliability requirements of software applications, and optimize the total power-consumption of the distribured systesm.

Software allocation is a well-researched area in the domain of embedded systems, including in hardware/software co-design \cite{Wolf2003ACodesign}, platform-based system design \cite{Sangiovanni-Vincentelli2004BenefitsDesign} and the Y-chart design approach \cite{ychart_Kienhuis2002}. It is a type of job-shop problem with constraints, and therefore finding an optimal solution, in the general case, is  NP-hard \cite{Fernandez-Baca1989AllocatingSystem}. The methods to solve such problems can be \textit{exact} or \textit{heuristic}. The exact methods, e.g., branch and bound, dynamic programming, etc., gurantee optimal solutions, neverthless, they are limited on large-scale problems \cite{Saidi2015AnArchitectures}. Moreover, applying exact methods on-non linear problems, which are prevalent in practice, is prohobitively expensive. Our previous work on solving the software allocation problem \cite{Mahmud5222}, we demonstrate the limitation of integer-linear programming (ILP) \cite{Bradley1977AppliedProgramming} using exact method by CPLEX solver . Similary, the scalability issues of exact methods on software allocation indicated in several research \cite{Saidi2015AnArchitectures}. In contrast, heuristic methods device a working technique to solve practical problems, which are usually large-scale, non-linear, without guranteeing optimality \cite{faragardi2018AECUs,Bucaioni2018MoVES:Systems}.  A particular type of heuristic is \textit{metaheuristics} which can be defined as ``an iterative generation process which guides a subordinate heuristic by combining intelligently different concepts for exploring and exploiting the search space, learning strategies are used to structure information in order to
find efficiently near-optimal solutions." \cite{Osman2005}.

Metaheuristics has found wide applications in many domains, e.g.,  deep learning, cellular networks, cloud computing, system partitioning, etc \cite{bibid}. Many of the existing meta-heuristic algorithms are nature inspired, e.g., genetic algorithm, evolutionary algorithms, simulated annehealing, ant colony, paticle-swarm optimization, etc. Applications of metaheuristics on the software allocation of real-time systems are in the early stages, neverthless, there exist some work, e.g., by Qin-Ma et al. \cite{bibid} on maximizing reliability of distributed computing sytems using hanybee algorithm, maximizing reliability of distributed systems using hill-climbing particle-swarm optimization by Yin et al. \cite{yin2007task}, etc. In this work, we apply differential evolution and hybrid particle-swarm optimization algorithms on a fault-tolerant distributed software applications that are developed using the AUTOSAR standard to optimize total power consumption of a distributed system. The software applications are refined by AUTOSAR runnables, which are schedulable pieces of programs, that execute periodically and with different sampling rates, also known as \textit{multirate}~\cite{Vinet2010APolynomials}. Furthermore, due to the different sampling rates that result in oversampling and undersampling effects, the timing analysis of signals propagation is complex \cite{mubeen2013support}. In order to maximize software applications reliability and meet user defined reliability goals, we consider replication of software components. The software applications consist of cause-effect chains that realize end-to-end functionality and have timing constrints, also known as end-to-end timing requirements. The applications are distributed over heterogeneous computing units that share a single network. In comparison to related works~\cite{Wozniak2013AnArchitectures, vsvogor2014extended,Saidi2015AnArchitectures}, 

The contributions of our work are summerized as follows: 
\begin{enumerate*}[label=(\roman*)]
	\item we provide a fitness function with constraints of the software allocation problem,
	\item we propose an approximation algorithm to reduce the overhead of computing the end-to-end delays of cause-effect chains,
	\item we provide performance comparison of the ILP method using CPLEX, differential evolution, and hybrid particle-swarm optimization algorithms with differential evolution, hill-climbing and stochastic hill-climbing algorithms
\end{enumerate*}. Our approach is evaluated on synthetic automotive applications that are generated according to the real-world automotive benchmark proposed by Kramer et al. \cite{Kramer2015RealFree}. In the evaluation, we show comparative performance of the various optimization algorithms interms of quality of solutions (or optimality), computation time,  and stability of the algorithms, for small and large software allocation problems. The tool applied in the evaluation is publicly accessible from BitBucket\footnote{\url{https://bitbucket.org/nasmdh/archsynapp/src/master/}}. 

The rest of the paper is organized as follows:
Section~\ref{sec_autosar} provides a brief overview of AUTOSAR, emphasizing on end-to-end timing and reliability modeling, and software allocation,
Section~\ref{sec_system} describes the AUTOSAR system model, including timing analysis, reliability and power-consumption assumptions.
Formulation of the software allocation problems is presented in Section~\ref{sec_problem}, which consist of the timing and reliability constraints and minimization of the total power consumption, followed by formulation of the optimization problem. 
We show how to solve the optimization problem using metaheuristics in Section~\ref{sec_solution}. The evaluation of our proposed methods are demonstrated in Section~\ref{sec_evaluation} using the automotive benchmark. Our work is compared to related works in Section~\ref{sec_RW}. Finally, we conclude the paper in Section~\ref{sec_conclusion}, and outline the possible future work.


