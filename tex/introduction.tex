\section{Introduction}
The software-to-hardware allocation is a very important step during the development of automotive embedded systems. Basically, it allows the designer to explore system-level solutions that meet functional and extra-functional software requirements together with resource availability on the execution platform. Software-allocation approaches for embedded systems are well-researched, including hardware/software co-design \cite{Wolf2003ACodesign}, platform-based system design \cite{Sangiovanni-Vincentelli2004BenefitsDesign} and the Y-chart design approach \cite{ychart_Kienhuis2002}. The problem is a type of bin-packing, and therefore finding an optimal solution, in the general case, is  NP-hard \cite{Fernandez-Baca1989AllocatingSystem}. The methods to solve such problems can be exact \cite{Saidi2015AnArchitectures}, which means solutions are guaranteed to be optimal, or heuristic which deliver near-optimal solutions \cite{faragardi2018AECUs}\cite{Bucaioni2018MoVES:Systems}. Exact methods such as Integer Linear Programming (ILP) \cite{Bradley1977AppliedProgramming} are shown to be efficient for applications with less number of allocatable components and allocator (e.g. processor) elements, assuming under different software application objectives and platform resources, hence they do not scale well for large systems. In contrast, heuristic methods can produce near-optimal solutions even for large systems, yet the exact methods are shown to return such solutions faster \cite{Wozniak2013AnArchitectures}. 
%In contrast to heuristic methods, the ILP-based solutions do not scale well for large and complex systems but return optimal solutions faster \cite{Wozniak2013AnArchitectures}. 
%Finding an optimal allocation of software to hardware solution with a minimal use of computational resources and power consumption is challenging in resource-constrained embedded systems.
Finding an optimal allocation solution with a minimal use of computational resources, such as processor, power consumption, memory, etc., is challenging. Even more so, when the distributed real-time application has to meet reliability and timing requirements. The latter complicate modeling and incur performance overheads regarding software allocation. In case of \textit{replication} \cite{Kopetz1989DistributedApproach}, which is a widely-used method for fault-tolerance, the allocation search space increases due to the replicas existing in the system. Moreover several automotive systems consist of chains of operating-system tasks, where the tasks in the chains execute periodically with different periods or rates, also called \textit{multi-rate systems} \cite{Wolf2012ComputersComponents}. At the communication level, such systems cause \textit{oversampling} and \textit{undersampling} effects that give rise to several timed paths, leading to different delay semantics. Furthermore, the software allocation is affected by the early selection of schedulability analysis technique to verify the timing requirements specified on the tasks as well as on the cause-effect chains that can be distributed over several nodes (computing units) in the system. 

In this paper, we propose an allocation scheme based on ILP for relatively small- and medium-sized fault-tolerant distributed applications, with the number of allocatable components not exceeding 15, operating-system tasks less than 100, and cause-effect chains in the range of 30 to 60. These parameters are deducted from the real-world automotive benchmark~\cite{Kramer2015RealFree}, previous experiences in developing automotive systems and experiments. The applications are distributed over heterogeneous computing units that share a single network. The scheme aims at minimizing the total power consumption of the system while meeting timing and reliability requirements. Our proposed solution targets the automotive domain, in particular the systems that conform to the AUTomotive Open System ARchitecture (AUTOSAR)\footnote{https://www.autosar.org/} standard. In comparison to the existing related works~\cite{Wozniak2013AnArchitectures,vsvogor2014extended,Saidi2015AnArchitectures}, we consider a fault-tolerant and multi-rate system model. Furthermore, we follow a highly integrated approach in the allocation process, which includes response-time analysis (RTA), and utilization bound checking (UB), as well as bounding the level of fault tolerance via the imposed reliability requirement on the application. The main contributions of our work are: i) an ILP model for the allocation of a fault-tolerant multi-rate application on heterogeneous nodes with the objective of minimizing the total power consumption, and ii) an approach for reducing overhead of replications and cause-effect chains on the allocation of such applications.

Our approach is evaluated on synthetic automotive applications that are generated using the \textit{real world automotive benchmark} proposed
%by Simon et al. 
\cite{Kramer2015RealFree}. In the evaluation, we show the performance of our proposed approach in terms of allocation time and resource efficiency for increasing size and complexity of applications. The tool and the synthetic applications used in this experiment are publicly available from BitBucket\footnote{Tool URL: https://bitbucket.org/nasmdh/archsynapp/src/master/}. 

The rest of this paper is organized as follows. Section~\ref{sec_autosar} provides a brief overview of AUTOSAR-based software development emphasizing the role of software allocation. Section~\ref{sec_system} describes the system model, followed by description of the timing, reliability, and power consumption models in Section~\ref{sec_extrafunc}. Section~\ref{sec_allocation} presents the proposed allocation scheme. Section~\ref{sec_evaluation} provides evaluation of the proposed approach using the automotive benchmark. Section~\ref{sec_RW} discusses the related work. Finally, Section~\ref{sec_conclusion} concludes the paper and outlines the possible future work.

