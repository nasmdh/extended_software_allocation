\section{Related Work}\label{sec_RW}
In a heterogeneous distributed system where computing nodes and communications links could have various failure rates, a reliability-aware allocation of tasks to nodes, and using links with the lowest failure rates can noticeably improve the system reliability~\cite{shatz1992task}\cite{kartik1997task}\cite{yin2007task}\cite{zhang2015maximizing}. Interleaving real-time constraints into the problem adds more complexity to reliability-aware task allocation in distributed systems~\cite{faragardi2013optimal}. As opposed to \cite{Wozniak2013AnArchitectures}\cite{Saidi2015AnArchitectures}, we assume that software applications are multirate, which increase the difficulty of software allocation due the complexity of their timing analysis, and increased search space as the result of increasing timed paths of cause-effect chains. Furthermore, we assume a fault-tolerant system model.

Although improving reliability of the system using a reliability-aware task allocation does not impose extra hardware/software cost,  in reliability-based design approach, redundancy (or replication) of software or hardware components is frequently applied to improve reliability. In such systems not only optimal allocation of software components (or replicas) should be taken into account but also the cardinality of the replicas should be limited for improved efficiency while meeting the desired reliability requirement. The integration of these two approaches (i.e., reliability-aware task allocation and application redundancy) is a promising technique to deal with high criticality of the system to fulfill the required reliability. For example, \cite{assayad2004bi} proposes a heuristic algorithm to maximize reliability of a distributed system using task replication while at the same time minimizing the makespan of the given task set. Furthermore, in systems with replication, it uses the Minimal Cut Sets method, which is an approximate algorithm, to calculate reliability of a system. In contrast, we apply an exact method based on state enumeration, which is applicable to the problem size assumed in this work.\vspace{-0.02cm}

In our problem, power consumption is the other criterion of the optimization problem. Several research work exist on improving power consumption in real-time distributed systems. The research work~\cite{bambagini2016energy} shows a survey of different methods on energy-aware scheduling of real-time systems, which categorizes the study into two major groups: i) Dynamic Voltage Scaling (DVS) ~\cite{devadas2012interplay}\cite{wang2015dynamic}, and ii) task consolidation to minimize the number of used computing and communication units~\cite{faragardi2013towards}, which is the approach followed in our work.\vspace{-0.02cm}

In the context of automotive systems, there are few works considering the reliability of a distributed system subject to real-time requirements of the automotive applications~\cite{islam2006dependability}\cite{kim2011autosar}. There are also other works discussing the allocation of software components onto nodes of a distributed real-time systems that consider other types of constraints other than reliability, for example, i) ~\cite{wang2004component} which considers computation, communication and memory resources, and ii)~\cite{vsvogor2014extended} which proposes a genetic algorithm for a multi-criteria allocation of software components onto heterogeneous nodes including CPUs, GPUs, and FPGAs. Our approach also considers a hetrogeneous platform, i.e., nodes with different power consumption, failure-rate, and processor speed. In this work, we consider only the processor time; however, it can easily be extended to take into account different types of memory consumption that the software applications require.
