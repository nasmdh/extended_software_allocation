\section{Conclusion and Future Work}\label{sec_conclusion}
Software to hardware allocation plays an important role in the development of distributed and safety-critical embedded systems. It is considered an NP-hard problem in the general case and a challenging task when applied to multi-rate systems where a software allocation operates on multiple time domains, hence causing undersampling and oversampling problems. In the allocation process, application level requirements need to be preserved while considering efficiency of computational resources such as processor, memory and power consumption.

In this work, we proposed an allocation scheme of a multi-rate real-time application that considers power consumption as the objective to be minimized, and timing and reliability requirements as constraints. The main contributions are an Integer Linear Programming model of the allocation problem, an algorithm for reducing replication overhead in the allocation process. Our approach is evaluated on synthetic automotive applications that are developed using the AUTOSAR standard and following the real-world automotive benchmark. Although we considered automotive applications for the evaluation, the proposed approach is equally applicable to resource-constrained embedded systems, especially with timing, power and reliability requirements, in any other domain that are developed using the principles of model-based development and component-based software engineering.

Our approach effectively applies to small and medium automotive applications. However, for complex applications, our approach is not scalable. Therefore, considering similar system models, we plan to extend the current work with scalable heuristic methods, e.g., evolutionary algorithms, genetic algorithms and simulated annealing.