\section{Conclusions and Future Work}\label{sec_conclusion}
Software to hardware allocation plays an important role in the development of distributed and safety-critical embedded systems. Effective software allocation ensures that high-level software requirements such as timing and reliability are satisfied, and design and hardware constraints are met after allocation. In fault-tolerant multirate systems, finding an optimal allocation of a distributed software application is challenging, mainly due to the complexity of cause-effect chains' timing analysis, as well as the calculation of software application reliability. The timing analysis is complex due to oversampling and undersampling effects, caused by the different sampling rates, and the complexity of the reliability calculation is caused by the interdependency of the computation nodes due to replicas. Consequently, the formulation of the problem, to find an optimal solution, becomes non trivial.

In this work, we propose an ILP model of the software allocation problem for fault-tolerant multirate systems. The objective function of the optimization problem is minimization of power consumption with the aim of satisfying timing and reliability requirements, and meeting design and hardware constraints. The optimization problem involves linearization of the reliability model with piecewise functions, formulating the timing model using logical constraints, and limiting the number of replicas that can be used in the allocation. Furthermore, the allocation consider two cases of timing analysis: response time analysis and utilization bound. 

Our approach is evaluated on synthetic automotive applications that are developed using the AUTOSAR standard, based on a real-world automotive benchmark. Although we consider automotive applications for the evaluation, the proposed approach is equally applicable to resource-constrained embedded systems, especially with timing, power and reliability requirements, in any other domain that are developed using the principles of model-based development and component-based software development. Our approach effectively applies to  medium-sized automotive applications, but does not scale for complex applications.  Considering similar system models, we plan to extend the current work with heuristic methods, e.g., genetic algorithms, simulated annealing, particle swarm optimization, etc., to handle large systems.

\subsubsection*{Acknowledgement}
This work is supported by the Swedish Governmental Agency for Innovation Systems (Vinnova) through the VeriSpec project, and the Swedish Knowledge Foundation (KKS) through the projects HERO and DPAC.