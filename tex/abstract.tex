\begin{abstract}
The growing complexity of automotive functionality has attracted revolutionary computing architectures such as mixed-criticality design, which enables effective consolidation of software applications with different criticality on a shared execution platform. Mixed-critical design that is required to satisfy end-to-end timing and reliability specifications should consider power-efficient software design in order to accommodate more and more functionality. Due to the recursive and exhaustive nature of the real-time and reliability analysis, exact methods, e.g., branch and bound, dynamic programming, are prohibitively expensive.

We propose hybrid particle-swarm optimization algorithms based on differential evolution and hill-climbing algorithms to minimize power consumption of the safety-critical software, which have end-to-end timing and reliability requirements, on a network of heterogeneous computing units. The optimization approach employs fault tolerance to maximize reliability of the software applications subsequently meet the reliability requirements. Our proposed integrated software-allocation approach is evaluated using a range of synthetic software applications based a real-world automotive benchmark. 

The evaluation makes comparative analysis of the differential evolution, particle-swarm optimization, integer-linear programming and hybrid particle-swarm optimization algorithms. The results show that the hybrid algorithms based on the hill-climbing algorithms outperform the rest of the meta-heuristic algorithms, in particular, the stochastic version of the hill-climbing algorithm scales well in large software allocation optimization problems while its overall optimality performance can be deemed acceptable.
\end{abstract}
