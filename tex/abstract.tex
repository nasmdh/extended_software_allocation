\begin{abstract}
% Software-to-hardware allocation plays an important role in the development of embedded systems, e.g., in hardware/software co-design, platform-based design, etc. In safety-critical systems, the execution of software applications must be predictable, event in the presence of faults. Therefore, the software application requirements such as timing and reliability should be checked during the allocation, at the early stage of system development. Moreover, the allocation should be efficient in order to accommodate more applications and their increasing demand for computation. For instance, due to the increasing automotive electronics, which are controlled by software programs, efficient allocation of software applications is highly needed, especially in modern cars and autonomous vehicles.
%Automotive embedded systems are often resource constrained.
Software-to-hardware allocation plays an important role in the development of resource-constrained automotive embedded systems that are required to meet timing, reliability and power requirements. 
% In safety-critical systems, the execution of software applications must be predictable, even in the presence of faults. 
% Therefore, the software application requirements such as timing and reliability should be checked during the allocation, at the early stage of system development. 
% Moreover, the allocation should be efficient in order to accommodate more applications and their increasing demand for computation. 
% For instance, due to the increasing automotive electronics, which are controlled by software programs, efficient allocation of software applications is highly needed, especially in modern cars and autonomous vehicles.
% In this paper, we propose an Integer Linear Programming optimization approach for the allocation of fault-tolerant software applications that are developed using the AUTOSAR standard. The allocation takes into account the timing and reliability constraints of the applications and the heterogeneity of the execution platform. The applications are multi-rate systems which consist of tasks and cause-effect chains that execute periodically, with different lengths. In the optimization, the objective is to minimize the total power consumption of the distributed system. Reducing power consumption enables consolidation of more applications and improves battery-life on the long run. We evaluate our approach on automotive domain, and evaluate its performance based on a real-world benchmark for different ranges of software applications. The results indicate that our software allocation approach, under the provided system model, effectively applies to small and medium software applications.
This paper proposes an Integer Linear Programming optimization approach for the allocation of fault-tolerant embedded software applications that are developed using the AUTOSAR standard. The allocation takes into account the timing and reliability requirements of the multi-rate cause-effect chains in these applications and the heterogeneity of their execution platforms. The optimization objective is to minimize the total power consumption of the applications that are distributed over more than one computing unit. 
The proposed approach is evaluated using a range of different software applications from the automotive domain, which are generated using the real-world automotive benchmark. The evaluation results indicate that the proposed allocation approach is effective and scalable while meeting the timing, reliability and power requirements in small- and medium-sized automotive software applications.
\end{abstract}
