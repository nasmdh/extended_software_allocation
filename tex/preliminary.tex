\section{AUTOSAR}\label{sec_autosar}
The AUTomotive Open System ARchitecture (AUTOSAR) partnership has defined the open standard AUTOSAR for automotive software architecture that enables manufacturers, suppliers, and tool developers to adopt shared development specifications, while allowing sufficient space for competitiveness. The specifications state standards and development methodologies on how to manage the growing complexity of Electronic/Electrical (E/E) systems, which take into account the flexibility of software development, portability of software applications, dependability, efficiency, etc., of automotive solutions. The conceptual separation of software applications from their infrastructure (or execution platform) is an important attribute of AUTOSAR and is realized through different functional abstractions \cite{NaumannAUTOSARBus}. 

\subsection{Software Application}
According to AUTOSAR, software applications are realized on different functional abstractions. The top-most functional abstraction, that is the Virtual Function Bus (VFB), defines a software application over a virtual communication bus using software components that communicate with each other via standard interfaces of various communication semantics. The behavior of a software component is realized by one or more atomic programs known as \textit{Runnables}, which are entities that are scheduled for execution by the operating system and provide abstraction to operating system tasks, essentially enabling behavioral analysis of a software application at the VFB level. The runnables are mapped to tasks, and subsequently are scheduled by the AUTOSAR operating system \cite{AUTOSAR2018Specification4.2.2}. The Runtime Time Environment (RTE), which is the lower-level abstraction, realizes the communication between Runnables via RTE Application Programming Interface (API) calls that respond to events, e.g., timing. Furthermore, the RTE implementation provides software components with the access to basic software services, e.g., communication, micro-controller and ECU abstractions, etc., which are defined in the Basic Software (BSW) abstraction \cite{NaumannAUTOSARBus}.
\subsection{Timing and Reliability of Applications}
The timing information of applications is a crucial input to the software allocation process. Among other extensions, the AUTOSAR Timing Extension specification \cite{AUTOSAR2017SpecificationExtensions} states the timing descriptions and constraints that can be imposed at the system-level via the \textit{SystemTiming} element. The timing constraints realize the timing requirements on the observable occurrence of events of type \textit{Timing Events}, e.g., Runnables execution time, and \textit{Event Chains}, also referred to as \textit{Cause-effect Chains} that denote the causal nature of the chain. In this work, we consider periodic events and cause-effect chains with different rates of execution (or activation patterns).

Although the importance of reliability is indicated in various AUTOSAR specifications via best practices, the lack of a comprehensive reliability design recommendations has opened an opportunity for flexible yet not standardized development approaches. In this paper, we consider application reliability as a user requirement and, in the allocation process, we aim at meeting the requirement via optimal placement and replication of software components.